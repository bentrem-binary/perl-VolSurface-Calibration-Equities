%% LyX 2.1.4 created this file.  For more info, see http://www.lyx.org/.
%% Do not edit unless you really know what you are doing.
\documentclass[english]{article}
\usepackage[utf8]{luainputenc}
\usepackage{amsmath}

\makeatletter

%%%%%%%%%%%%%%%%%%%%%%%%%%%%%% LyX specific LaTeX commands.
\newcommand{\noun}[1]{\textsc{#1}}

\makeatother

\usepackage{babel}
\begin{document}

\title{Volatility Surface Construction }

\maketitle

\section*{Introduction }

Binary.com uses a modified Stochastic alpha, beta, rho (SABR) model,
a stochastic volatility (vol) model, to estimate the volatility smile
in the derivatives market. For a more general overview of the general
SABR model, please refer to (Clark, 2011). 

An expression for the market-implied Black-Scholes-Merton volatility
$\sigma_{X}(K,T)$ at strike $K$ and expiry $T$, following Clark's
(2011) terminology, can be written as
\begin{equation}
\sigma_{X}(K,T)=\frac{\alpha z}{\chi(z)}\left[1+\left(\frac{\rho\nu\alpha}{4}+\frac{2-3\rho^{2}}{24}\nu^{2}\right)T\right]\label{eq:SABR}
\end{equation}


where $\alpha$ is at-the-money (ATM) volatility, $\nu$ is volatility
of volatility, $\rho$ is correlation between spot and volatility,
and we use
\begin{equation}
z(K,T)=\frac{\nu r}{\alpha}\tanh\left(\frac{\log\left(\frac{F_{0,T}}{K}\right)}{r}\right)\sqrt{\frac{t_{0}}{T}},\label{eq:z}
\end{equation}
 where $F_{0,T}$ is ATM spot, $t_{0}=1$ yr, 
\[
\chi(z)=\log\left(\frac{\sqrt{1-2\rho z+z^{2}}+z-\rho}{1-\rho}\right).
\]
The parameter $r$ above is set as
\[
r=\begin{cases}
\log\left(s_{+}\right) & \text{if }F_{0,T}>K\\
\log\left(s_{-}\right) & \text{if }F_{0,T}\le K
\end{cases},
\]
where $s_{+},s_{-}$ are constants that represent maximum and minimum
desired strike ratios. Also, we have used used the stochastic normal
{[}Clark's (2011) $\beta=1${]} assumption as opposed to the stochastic
lognormal ($\beta=0$) or more general hybrid ($0<\beta<1$) implementations.
In this respect, we could say we use the SAR (Stochastic alpha and
rho) model. Note that the time $t_{0}$ in Eq. \ref{eq:z} is required
for dimensional consistency and either should to be inserted in production
code or removed everywhere with $T$ always specified in years. In
the standard SAR model (Eq.\ref{eq:SABR}) the dependence on maturity,
or term structure, comes from the factor in the square brackets with
other factors remaining static. Our implementation, however, discards
the square-bracket part and asserts maturity dependence with $\alpha(T)$
and $z(K,T)$. Binary.com's volatility surface is
\[
\sigma(K,T)=\frac{\alpha(T)\,z(K,T)}{\chi(z)}
\]



\subsection*{Definitions }
\begin{itemize}
\item \noun{Moneyness} is always calculated with respect to spot $F_{0}$.
\[
\text{Moneyness}=K/F_{0}.
\]

\item \noun{Correlation} denotes correlation between stock returns and implied
volatility of the options. This is not simply the statistical correl
function in excel. It is calculated using some other function based
on kurtosis and skew. Discussed later. 
\item \noun{Skew} is Implied Volatility of 90\% Strike minus Implied Volatility
of 100\% Strike. This is scaled by square root of maturity. 
\end{itemize}

\section*{Overview }

Any distribution can be visibly perturbed using ATM Vol, Skew \& Kurtosis.
This model accepts these parameters and some other finer parameters
to generate an extremely smooth surface. 

When skew is very high it means OTM puts are priced at high premiums
compared to ATM. Likewise if skew is very low OTM puts are priced
at low premiums compared to ATM. When skew increases from low to high,
what is the behavior of correlation between stock \& volatility? 

\emph{When the market is falling drastically and volatility is increasing
this implies high Correlation. When the market is stable, and volatility
is low implies low Correlation}. Hence it can be said that the correlation
between an index and volatility increases as negative skew increases.
Kurtosis signifies that the volatility is clustered around the mean
or that the distribution of ATM vol for different maturities will
be similar. So high kurtosis means vols are clustered in the middle
of the distribution and index movements will not cause much change
in volatility. Hence lower correlation. 

Similarly when low Kurtosis in volatility signifies that volatility
is very dispersed or that small changes in the index might cause higher
volatility changes. Hence for this case correlation between the index
and volatility is high. So Kurtosis High implies Correlation Low,
Kurtosis Low implies Correlation High. 

Some quants might question this approach and say “Why not just use
a simple correl function to calculate correlation between index returns
and volatility?”. In a perfect world this should be done but since
the correlation value obtained using the correl function (assuming
excel) is unstable, this is not a practical approach. This leads to
unstable calibration parameters from one day to another. 


\section*{Functional Forms }


\subsection*{ATM Volatility and Skew }

The calibration approach is based upon modeling the term structure
of ATM Volatility and ATM Skew using exponential functions. It is
widely observed that any ATM Vol term structure or skew term structure
is convex (mostly) and rarely concave. An exponential function is
in general a convex function. 

We try to study this using the functional form: 
\[
w_{1}e^{x_{1}}-w_{2}e^{x_{2}}
\]
The above form allows us to create both concave and convex functional
forms or curves that are continuously differentiable. With appropriate
scaling we can model the ATM term structure and skew based on the
above equation. This will become our base function for calibrating
ATM Vol and Skew term structure. 


\subsection*{Volatility-Surface Model Parameters }

We use the following parameters for our calibration:

$\begin{array}{ccc}
\text{\text{Parameter}} & \text{PERL Identifier} & \text{Dimensions}\\
a_{1} & \text{atmvolshort} & \text{T}^{-1/2}\\
a_{2} & \text{atmvol1year} & \text{T}^{-1/2}\\
a_{3} & \text{atmvolLong} & \text{T}^{-1/2}\\
a_{4} & \text{atmWingL} & \text{T}^{-1}\\
a_{5} & \text{atmWingR} & \text{T}^{-1}\\
a_{6} & \text{skewshort} & \text{T}^{-1/2}\\
a_{7} & \text{skew1year} & \text{T}^{-1/2}\\
a_{8} & \text{skewlong} & \text{T}^{-1/2}\\
a_{9} & \text{skewwingL} & \text{T}^{-1}\\
a_{10} & \text{skewwingR} & \text{T}^{-1}\\
a_{11} & \text{kurtosisshort} & \text{T}^{-1/2}\\
a_{12} & \text{kurtosislong} & \text{T}^{-1/2}\\
a_{13} & \text{kurtosisgrowth} & \text{T}^{-1}\\
a_{14} & \text{strikeDn} & *\\
a_{15} & \text{strikeUp} & *
\end{array}$

{*} dimensionless

For ATM vol we use
\[
\alpha^{2}(T)=a_{3}^{2}+\frac{\left(a_{2}^{2}-a_{3}^{2}\right)\left(e^{-a_{5}T}-e^{-a_{4}T}\right)+\left(a_{1}^{2}-a_{3}^{2}\right)\left(e^{-a_{5}t_{0}-a_{4}T}-e^{-a_{4}t_{0}-a_{5}T}\right)}{e^{-a_{5}t_{0}}-e^{-a_{4}t_{0}}}
\]
Note that we are scaling variance with the exponential functions not
the actual vols. This is as per the rationale that variance is scalable
and can be linearly interpolated but volatility is the square root
of variance so it is not easy to interpolate. Also variances are additive.
The exponential function parameters $a_{4}=$atmWingL and $a_{5}=$atmWingR
control the behavior of the vol term structure and have units of inverse
time. If atmWL is greater than atmWR then the left side of the curve
is lifted up, while the right side decreases. This effect is necessary
for keeping the term structure smooth and preserving the shape of
the smile on both sides of the ATM vol. We may also check that $\alpha(0)=a_{1}$,
$\alpha\left(t_{0}\right)=a_{2}$, and $\lim_{T\rightarrow\infty}\alpha(T)=a_{3}$.
Presumably this explains the naming of these parameters, but be cautious
not to regard these as any meaningful volatility values.

For volatility skew ($\gamma)$ we use

\[
\gamma(T)=a_{8}+\frac{\left(a_{7}-a_{8}\right)\left(e^{-a_{10}T}-e^{-a_{9}T}\right)+\left(a_{6}-a_{8}\right)\left(e^{-a_{10}t_{0}-a_{9}T}-e^{-a_{9}t_{0}-a_{10}T}\right)}{e^{-a_{10}t_{0}}-e^{-a_{9}t_{0}}}.
\]
 Here we find that $\gamma(0)=a_{6}$, $\gamma\left(t_{0}\right)=a_{7}$,
and $\lim_{T\rightarrow\infty}\alpha(T)=a_{8}$.

For volatility of volatility, we use

\[
\nu=\frac{2\gamma(T)}{\rho(T)},
\]
where correlation $\rho$ is described below.
\[
\rho(T)=-\left[\frac{3}{2}\left(1+\frac{k\alpha}{\gamma^{2}}\right)\right]^{-\frac{1}{2}}
\]
 and volatility kurtosis $k$ is
\[
k(T)=a_{12}+\left(a_{11}-a_{12}\right)e^{-a_{13}T}.
\]
 As with $\alpha$ and $\gamma$, we can check for certain maturities,
$k(0)=a_{11}$ and $\lim_{T\rightarrow\infty}k(T)=a_{12}$.

Example : 

atmvolshort 0.13 

atmvol1year 0.19 

atmvolLong 0.12 

atmWingL 1.27 

atmWingR 1.56 

In the example above these 5 points are mentioned in the same order.
The two wings are basically the weights to stretch the ATM term structure
on either end. As an input the exponential function doesn't take the
volatilities but variances since variances can be linearly interpolated. 

And similarly for skew exactly the same logic is applied for skew
parameters. 
\begin{enumerate}
\item Short term skew 
\item 1 Year skew 
\item Longest term skew 
\end{enumerate}
The ATM Vol term structure generated by the functional form will match
the surface's ATM Vols approximately. The skew term structure similarly
calculated from the functional form will match the skew calculated
from the surface. 


\subsection*{Skew Params \& Values }

Example: 

skewshort -27\% 

skew1year -8\% 

skewlong -3\% 

skewwingL 2.01 

skewwingR 2.03 

All this generates the skew and atm vols only (not the surface). 


\subsection*{Functional Relationships}

Skew is asymmetric. When skew increases the left side shifts up and
the right side shifts down. When skew decrease similarly the left
side shifts down, and the right side shifts up. Kurtosis on the other
hand provides a symmetric control over the wings of a surface. Kurtosis
basically shifts the wings of the curve in a symetric way. We see
that
\[
k\text{ is large}\implies\rho\text{ is small}
\]
and
\[
k\text{ is small}\implies\rho\text{ is large}.
\]


Similar logic applies to skew. When skew is higher it means OTM options
are priced higher and might result in volatility increasing even for
a slight fall in the market. So Spot movement is very much correlated
with volatility when skew is high. 

Sabr Tanh Parameters (Flattening): 

strikeUp 120\% 

strikeDown 80\% 

Two extra params for control are strikeDn and strikeUp. These are
the two strike limits beyond which the curvature effect is replaced
by a flattening or flattening which occur beyond the designated strikes. 


\section*{Volatility Calculation }

As discussed previously, Eq.\ref{eq:z} contains the tanh function.
The limits of tanh are between -1 and +1, and since it is symmetric
it is well suited for volatility surface modeling. After building
the ATM and skew term structure the other strikes are weighted based
on moneyness. The function below builds the surface. 

Notation: Moneyness is calculated with respect to spot. 
\[
x=\frac{\log\left(\frac{K}{F_{0,T}}\right)}{\alpha}\sqrt{\frac{t_{0}}{T}}
\]
 So we basically scale moneyness with the tanh function. When strike
is very near to ATM, the computation is numerically unstable, so we
perform the following test: 

\texttt{If Abs(x) \textless{} 0.0000000001 Then }

\qquad{}\texttt{sabrVol2 = atmvol} 

\qquad{}\texttt{Exit Function} 

\texttt{End If} 

The tanh function is used to extend the curve between the ATM and
the end points. The end point on one end is StrikeDn and for the other
end is StrikeUp. It's basically a sort of trigonometric interpolation
between two moneyness levels. 
\begin{enumerate}
\item The ATM and the Upward extremum on one side 
\item The ATM and the Downward extremum on other side 
\end{enumerate}
\texttt{If (x \textgreater{} 0) Then }

\texttt{x = xmax {*} tanh(x / xmax) }

\texttt{Else }

\texttt{x = xmin {*} tanh(x / xmin) }

\texttt{End If }

Since the functional form of the volatility as a whole is everywhere
differentiable, it is continuous. Hence we can always get local volatility
for any of the spot and time points. Second derivative problem never
happens because the dependence on $K$ is only through exponentials. 


\subsection*{Optimization }

We use a form of the Downhill Simplex Method or Nelder-Mead (available
as the R function optim). This can also be coded in other languages.
In the excel solver we can specify constraints using variable conditions.
Here the function which is specified as parameter applies the constraints.
For instance if I don't want ATM vol to go above 0.5, I will return
a residual value of 1000 or any other large number. Thus, the optimization
function that is calling my function will understand that ATM vol
should not go above 0.5 (or 50 percent). It basically minimizes the
function by assuming a simplex (or N vertices polygon). It is the
shape by reflection, expansion, contraction and reduction operations
on the geometrical figure. This results in an excellent optimization
in few steps. A major problem with the above algorithm is that it
will stop at the local minima. Given the large number of fitted parameters
and nonlinear implied volatility function, using a global optimization
technique such as simulated annealing or basin hopping should be used.


\section*{Validation}

There are 8 checks we use to make sure we have a valid volatility
surface.
\begin{enumerate}
\item Inputs
\item Identical Surface
\item Volatility Jump
\item Spot Reference
\item Age
\item Smile
\item Term Structure
\item Smile Admissibility
\end{enumerate}
Checks 1 through 6 consist of a number of sanity verifications. Check
7 aims to eliminate arbitrage possibilities across tenor and check
8 prevents arbitrage across strikes. The following describes these
final two checks.


\subsection*{Term Structure and Smile Admissibility}

As necessary and sufficient conditions to prevent arbitrage possibilities
(Roper, 2010), the implied European call price surface $C(K,T)$ must
have the following properties.
\begin{enumerate}
\item $\frac{\partial C(K,T)}{\partial T}\ge0.$ We check that for any consecutive
call prices ordered by maturity at a given strike, the second price
is less than or equal to the first. (Calendar arbitrage check)
\item $\frac{\partial^{2}C(K,T)}{\partial K^{2}}\ge0$. We check that implied
European call prices for any 3 consecutive strike points, the middle
strike value is less than or equal to the average of the sum of the
values of the end points. (Admissibility Check \#2)
\item $C(K,T)\ge\max(S-K,0)$. 
\item $C(K,0)=\max(S-K,0)$.
\item $\lim_{K\rightarrow\infty}C(K,T)=0$. We check that the implied call
price for large $K$ is close to zero. (Admissibility Check \#5)
\end{enumerate}
Because we are interested in digital options rather than European-style
options, we use the following checks in place of (3) and (4) in the
list above.
\begin{itemize}
\item $\frac{\partial C(K,T)}{\partial K}\le0$. We check that for any consecutive
strike points, the second price is less than or equal to the first.
(Admissibility Check \#1)
\item $\lim_{K\rightarrow0}C_{\text{dig}}(K,T)=1$. We check that the implied
digital call price for small $K$ is close to unity. (Admissibility
Check \#4)\end{itemize}

\end{document}
